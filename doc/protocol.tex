\documentclass{scrartcl}

\usepackage{charter}
\usepackage{amsmath}

\title{Pycc Protokoll Specification 0.1}
\author{Christoph Sterz}

\maketitle

\begin{document}
	\section{Preface}
	The pycc protocol is used to send messages and commands between the various pycc clients. It provides a simple way to format given 
	byte-strings to package- and client-information and the actual message.

	\section{Specification}
	\subsection{New connection and protocol handshake}
	A connection has 3 statuses:
	\begin{itemize}
		\item \textbf{new:} \quad Connection is new and not even received by the connection partner.
		\item \textbf{init:} \quad	Connection is established but needs to be configured.
		\item \textbf{open:} \quad	Connection is established.
	\end{itemize}
	
	Example \textbf{new} Socket Packet:	\begin{center}{$PYCC \mid [versions] \mid PyCC-Node$}\end{center}
	\begin{itemize}
		\item	Protocol begins with \quad $PYCC$
		\item	Single Protocol parts are seperated by pipes \quad $\mid$
		\item	$[versions]$ contains the protocol-versions (comma seperated)  used for this connection.
		\item $PyCC-Node$ describes additional information about the sender (not imprtant).
	\end{itemize}

	When the socket receives a package formatted that way it sends a packacke of the same format. $(PYCC|[version]|PyCC-Node)$
	Finally an $OK,[version]$ Message is sent back to confirm the Connection. 

	\newpage

	\subsection{Message Specification}
	Example:	\begin{center}{$A[comHandle]:[Message]$}\end{center}
	Where:
	\begin{itemize}
	
		\item \textbf{First Letter}: \quad indicates the type of message
			\subitem \textbf{A}: \quad is a Request Message
			\subitem \textbf{O}: \quad is the Answer to that Request
			\subitem \textbf{E}: \quad is the Errormessage raised
		\item \textbf{[comHandle]} \quad is a number indicating the connection count Client has odd counts, Server evens.
		\item :\textbf{Message} A Message preceeded by : .

	\end{itemize}
	\section{Exceptions}
	If a protocoll format \emph{does not} provide the features that are demanded in the specification a ProtocolException is raised.
	May be useful sometime.
\end{document}
