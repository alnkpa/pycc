\documentclass{scrartcl}

\usepackage{charter}
\usepackage{amsmath}

\title{Pycc Protokoll Specification 0.1}
\author{Christoph Sterz}

\maketitle

\begin{document}
	\section{Preface}
	The pycc protocol is used to send messages and commands between the various pycc clients. It provides a simple way to format given 
	byte-strings to package- and client-information and the actual message.

	\section{Specification}
	\subsection{New connection and protocol handshake}
	A connection has 3 statuses:
	\begin{itemize}
		\item \textbf{new:} \quad Connection is new and not even received by the connection partner.
		\item \textbf{init:} \quad	Connection is established but needs to be configured.
		\item \textbf{open:} \quad	Connection is established.
	\end{itemize}
	
	Example \textbf{new} Socket Packet:	\begin{center}{$PYCC \mid [version] \mid Client$}\end{center}
	\begin{itemize}
		\item	Protocol begins with \quad $PYCC$
		\item	Single Protocol parts are seperated by pipes \quad $\mid$
		\item	$[version]$ contains the protocol-version used for this connection.
		\item $Client$ describes the PYCC-Node the package is adressed to.
	\end{itemize}

When the socket receives a package formatted that way it sends a packacke of the same format $(PYCC|[version]|Client)$

	\subsection{Message Specification}
	Example:	\begin{center}{$$}\end{center}

	\section{Exceptions}
	If a protocoll format \emph{does not} provide the features that are demanded in the specification a ProtocolException is raised.
\end{document}
